\documentclass[11pt, a4paper]{article}
\usepackage[utf8]{inputenc}
\usepackage[czech]{babel}
\usepackage[left=15mm,text={18cm,25cm},top=25mm]{geometry}
\usepackage[IL2]{fontenc}
\usepackage{times}
\usepackage{amsthm}
\usepackage{amssymb}
\usepackage{amsmath}

\usepackage{hyperref}

\theoremstyle{definition}
\newtheorem{myDef}{Definice}

\theoremstyle{plain}
\newtheorem{myLem}{Věta}

\title{}
\author{}
\date{}

\begin{document}

\begin{titlepage}
    \begin{center}
    \Huge
    \textsc{Fakulta informačních technologií \\[0.4em] Vysoké učení technické v~Brně}\\
    \vspace{\stretch{0.382}}
    \LARGE Ukládání a příprava dat\,–-\,projekt\\[0.3em]
    Zdravotnictví v ČR\\
    \vspace{\stretch{0.618}}
    \end{center}
    {\Large 2021 \hfill Jakub Kočalka (xkocal00), Jiří Žilka (xzilka11)}
\end{titlepage}

\section*{Úvod}

Z~nabízených témat bylo vybráno téma číslo 2: Zdravotnictví v~ČR. Cílem projektu je seznámit se se zpracováním rozsáhlých/nestrukturovaných dat, seznámit se s~přípravou těchto dat pro další využití (např. pro získávání znalostí z~databází) a~tvorbou popisných charakteristik pro zvolená data.

\section{Návrh zpracování a uložení dat}
Tato část projektu se zabývá seznámením se s vybraným tématem, analýzou dat z nabízených zdrojů, návrhu způsobu získání datových sad z daných zdrojů, výběru vhodné NoSQL databáze a uložení dat do zvolené databáze.

Byla zkoumána data ze zadaných zdrojů (\href{https://nrpzs.uzis.cz/index.php?pg=home--download&archiv=sluzby}{Národní registr poskytovatelů zdravotních služeb} a \href{https://www.czso.cz/csu/czso/obyvatelstvo-podle-petiletych-vekovych-skupin-a-pohlavi-v-krajich-a-okresech}{data Českého statistického úřadu o obyvatelstvu ČR}). Na základě analýzy datových sad a jejich struktury byla pro jejich uložení zvolena NoSQL databáze MongoDB. Data nejsou tvořena časovými řadami, takže nevyhovuje InfluxDB, ani se nehodí uložení dat pomocí grafové databáze Neo4j. 

\subsection{Načtení dat ze zdrojů}
Pro získání dat ze zadaných zdrojů byl vytvořen skript \texttt{download.py}. Tento skript také nahrává data do cloudové databáze. 

Pro většinu zadaných dotazů jsou potřeba pouze aktuální data, ale pro druhý dotaz skupiny A je potřeba získat i historii poskytovatelů zdravotních služeb. Při práci z daty bylo zjištěno, že historická data jsou velice objemná a je problematické je uložit do použité cloudové databáze. Tento problém byl aktuálně vyřešen načtením pouze nejnovějšího datového souboru (historie by se pak určovala pomocí údaje \texttt{DatumZahajeniCinnosti}). Alternativně by mohl být problém vyřešen redukcí historických dat na nezbytně nutné položky nebo ukládáním pouze souhrných hodnot pro každý kraj v daném čtvrtletí.

\subsection{Příprava dat}
Ze zkoumání zadaných dotazů a struktury dat vyplynulo, že některé atributy dat nebudou použity a tedy je možné je vypustit. Pro první dotaz \textbf{skupiny A} jsou v záznamech potřebné pouze údaje o kraji a okresu, oboru péče. Pro vyhodnocení dotazu jsou poutřebné pouze aktuální data.

Druhý dotaz \textbf{skupiny A} využívá pro práci z daty pouze položku obor péče. Jako jediný zpracovává ale i historické údaje, proto je pro něj potřeba načíst i~starší datové soubory o poskytovatelích zdravotních služeb. Protože využívá pouze položku obor péče, bylo by možné historické záznamy ukládat pouze jako trojice ("\texttt{ZdravotnickeZarizeniId}", "\texttt{OborPece}", "datum vytvoření datového souboru, do kterého záznam patří") nebo rovnou jako celkové počty poskytovatelů oborů pro každé čtvrtletí.

Dotazy \textbf{skupiny B} využívají dat z obou zadaných zdrojů. První dotaz bude vždy vytvářet součty pro každý kraj. Z dat o obyvatelstvu se použijí záznamy vyjadřující celkový počet mužů/žen ve věkových kategoriích 20 let a více kraji. Záznamy o poskytovatelích se seskupí podle položky \texttt{KrajKod}, potom se vyberou záznamy podle položky \texttt{OborPece} a výsledkem budou počty záznamů ve skupinách. Dotaz bude vyhodnocovat poměr těchto počtů pro každý kraj.

Druhý dotaz bude také zpracovávat údaje o obyvatelstvu podle věkové kategorie a kraje. Dále bude využívat záznamy o celkovém počtu obyvatel v kraji. Záznamy o poskytovatelích zdravotních služeb bude vybírat podle položek \texttt{DruhZarizeniKod}, \texttt{OborPece} a \texttt{KrajKod}.

Výsledky těchto dotazů se budou zobrazovat jako hodnoty souhrných počtů v grafech, a proto jsou některé další informace zbytěčné. Skript proto po načtení hodnot ze zdrojů odstraňuje nevyužívané atributy před nahráním dat do databáze.

Dotazy \textbf{skupiny C} budou využívat údaje počtu obyvatel v městech podle věkových skupin. Údaje o poskytovatelích rozdělí podle obcí a následně podle oborů péče.

\subsection{Uložení dat do databáze}
Skript \texttt{download.py} vytvoří Mongo klienta pro připojení ke cloudové databázi a následně vytvoří databáze. Connection string se získává z proměnné prostředí (environment variable). Připravená data z obou zadaných zdrojů nahrává do databáze, pokud není použit argument \texttt{--local}. 

\section{Implementovaný systém pro získání, ukládání a zpracování dat}

\end{document}
